\documentclass{mathproblems}
\usepackage[all]{xy}
\usepackage{stmaryrd}
\usepackage{amsmath}
\usepackage{amssymb,mathrsfs}
\usepackage{amsthm}
\usepackage{xcolor}
\usepackage{multirow}
\usepackage{tikz}
\usepackage{tikz-cd}
\usetikzlibrary{cd,shapes,arrows,positioning,calc}

\usepackage{enumerate}
\usepackage{mathrsfs}
\usepackage{enumitem}
\usepackage{hyperref}
\usepackage{bm}
\usepackage{pgfplots}

\usepackage{geometry}
\geometry{left=2.5cm,right=2.5cm,bottom=3.4cm}


\newcommand\V{\mathcal{V}}
\newcommand\tD{\tilde{\mathcal{D}}}
\newcommand\Q{\mathbb{Q}}
\newcommand\bS{\mathbb{S}}
\newcommand\cH{\mathcal{H}}
\newcommand\FF{\mathscr{F}}
\newcommand\cF{\mathcal{F}}
\newcommand\A{\mathcal{A}}
\newcommand\B{\mathcal{B}}
\newcommand\R{\mathbb{R}}
\newcommand\C{\mathbb{C}}

\newcommand\bD{\mathbb{D}}
\newcommand\Fil{\mathrm{Fil}}
\newcommand\pDiv{p\mathsf{Div}}

\newcommand\T{\mathcal{T}}
\newcommand\Z{\mathbb{Z}}
\newcommand\N{\mathbb{N}}
\newcommand\NN{\mathcal{N}}
\newcommand\PP{\mathcal{P}}
\newcommand\QQ{\mathfrak{Q}}
\newcommand\D{\mathcal{D}}
\newcommand\OO{\mathcal{O}}
\newcommand\F{\mathbb{F}}

\newcommand\p{\mathfrak{p}}
\newcommand\g{\mathfrak{g}}
\newcommand\h{\mathfrak{h}}
\newcommand\q{\mathfrak{q}}
\newcommand\fC{\mathfrak{C}}
\newcommand\fa{\mathfrak{a}}
\newcommand\fb{\mathfrak{b}}
\newcommand\fc{\mathfrak{c}}
\newcommand\fd{\mathfrak{d}}
\newcommand\m{\mathfrak{m}}
\newcommand\al{\alpha}
\newcommand\pel{(\mathcal{O},\ast,\Lambda,\langle\cdot,\cdot\rangle,h)}
\newcommand\lm{\Lambda}
\newcommand\om{\Omega}
\newcommand\gm{\Gamma}
\newcommand\hZ{\widehat{\Z}}
\newcommand\LL{\mathcal{L}}
\newcommand\Crys{\mathsf{Crys}}
\newcommand\dR{\mathrm{dR}}
\newcommand\Qb{\overline{\mathbb{Q}}}
\newcommand\Qpb{\overline{\mathbb{Q}}_{p}}
\newcommand\Qp{\mathbb{Q}_{p}}
\newcommand\Qlb{\overline{\mathbb{Q}}_{\ell}}
\newcommand\Fb{\overline{\mathbb{F}}}
\newcommand\Fpb{\overline{\mathbb{F}}_{p}}
\newcommand\Fqb{\overline{\mathbb{F}}_{q}}
\newcommand\Zb{\overline{\mathbb{Z}}}
\newcommand\rb{\overline{\rho}}
\newcommand\der{\mathrm{der}}
\newcommand\GO{\mathrm{GO}}
\newcommand\GU{\mathrm{GU}}
\newcommand\GL{\mathrm{GL}}
\newcommand\Frob{\mathrm{Frob}}

\newcommand\alg{\mathrm{alg}}
\newcommand\un{\mathrm{un}}
\newcommand\PSL{\mathrm{PSL}}
\newcommand\PSO{\mathrm{PSO}}
\newcommand\PGL{\mathrm{PGL}}
\newcommand\SL{\mathrm{SL}}
\newcommand\SU{\mathrm{SU}}
\newcommand\SO{\mathrm{SO}}
\newcommand\Sp{\mathrm{Sp}}
\newcommand\GSp{\mathrm{GSp}}
\newcommand\Spin{\mathrm{Spin}}
\newcommand\simto{\stackrel{\sim}{\longrightarrow}}
\newcommand\simeqto{\stackrel{\simeq}{\longrightarrow}}
\newcommand\congto{\stackrel{\cong}{\longrightarrow}}
\renewcommand\P{\mathbb{P}}



\newcommand\Rep{\mathsf{Rep}}
\newcommand\Sch{\mathsf{Sch}}
\newcommand\PR{\mathrm{PR}}
\newcommand\DP{\mathrm{DP}}
\newcommand\Ra{\mathrm{Ra}}
\newcommand\Hasse{\mathrm{Hasse}}
\newcommand\cris{\mathrm{cris}}
\newcommand\ad{\mathrm{ad}}
\newcommand\Ad{\mathrm{Ad}}
\newcommand\univ{\mathrm{univ}}
\DeclareMathOperator{\Ext}{Ext}
\DeclareMathOperator{\Spl}{Spl}
\DeclareMathOperator{\Tr}{Tr}
\DeclareMathOperator{\tr}{tr}
\DeclareMathOperator{\Cl}{Cl}
\DeclareMathOperator{\Pic}{Pic}
\DeclareMathOperator{\rad}{Rad}
\DeclareMathOperator{\Hom}{Hom}
\DeclareMathOperator{\Spec}{Spec}
\DeclareMathOperator{\Aut}{Aut}
\DeclareMathOperator{\Nm}{Nm}
\DeclareMathOperator{\Ann}{Ann}
\DeclareMathOperator{\St}{St}
\DeclareMathOperator{\ev}{ev}

\DeclareMathOperator{\Ker}{Ker}
\DeclareMathOperator{\Coker}{Coker}
\DeclareMathOperator{\im}{im}

\DeclareMathOperator{\Id}{Id}
\DeclareMathOperator{\Ind}{Ind}
\DeclareMathOperator{\Res}{Res}
\DeclareMathOperator{\Gal}{Gal}
\DeclareMathOperator{\End}{End}
\DeclareMathOperator{\ord}{ord}
\DeclareMathOperator{\Sym}{Sym}
\DeclareMathOperator{\disc}{disc}
\DeclareMathOperator{\rank}{rank}
\DeclareMathOperator{\Lie}{Lie}

\DeclareMathOperator{\Sht}{Sht}
\DeclareMathOperator{\Bun}{Bun}

\DeclareMathOperator{\Gr}{Gr}

\course{Interview Questions on Algebra}
\studentname{Wenhan Dai}



\begin{document}

\centerline {\textbf{Set IV: Normal Forms}}

\begin{questions}

\miquestion {\color{blue} What is the connection between the structure theorem for modules over a PID and conjugacy classes in the general linear group over a field?}

\textit{Statement.} (aka fundamental theorem of finitely generated modules over a PID) Let $R$ be a PID. For any finitely generated $R$-module $M$, there exists a sequence of (pricipal) $R$-ideals, say $(d_1)\supset (d_2) \supset \cdots \supset (d_n)$ (or equivalently, $d_1 \mid d_2 \mid \cdots \mid d_n$) such that
$$
M\cong \bigoplus_{i=1}^n R/(d_i).
$$

\textit{Comment.} This is a general version of the following. For vector spaces over fields, they are all generated by a set of basis flatly. However, the modules are possibly not flat, but they can be realized as quotients over PIDs.

\textit{Answer.} Two elements of $\operatorname{GL}_n(F)$ are conjugate if and only if they obtain the same {\color{violet}\textbf{rational canonical form}} over $F$. The proof of this fact essentially relies on the structure theorem of finitely generated modules over a PID.

\miquestion {\color{blue} Explain how the structure theorem for finitely-generated modules over a PID applies to a linear operator on a finite dimensional vector space.}

\textit{Answer.} Given $T: V\to V$ as supposed for $V$ over $F$. Let $V^T=V$ set-theoretically. Then $V^T$ is an $F[x]$-module that admits actions by $f(T):V \to V$ with $f\in F[x]$. By the structure theorem,
$$
V^T=\bigoplus_{i=1}^m F[x]/(f_i)
$$
with $m\leq \dim V=n$, $f_1\mid \cdots \mid f_m$ for $f_i\in F[x]$. Using Chinese remainder theorem, we see $f_1\cdots f_m$ is the characteristic polynomial of $T\in \operatorname{GL}_n(F)$.

\miquestion {\color{blue} I give you two matrices over a field. How would you tell if they are conjugate or not? What theorem are you using? State it. How does it apply to this situation? Why is $k[T]$ a PID? If two matrices are conjugate over the algebraic closure of a field, does that mean that they are conjugate over the base field too?}

\textit{Answer.} The first several questions are solved by Question 1 and 2.

It turns out that $k[T]$ is ED via the norm $N=\deg: k[T]\backslash \{0\} \to \mathbb{N}$ (note: here comes the reason why we don't define $\deg(0)$).

Yes, since the rational canonical form does not depend on any property of fields.

\miquestion {\color{blue} If two real matrices are conjugate in $\textrm{M}_n(\mathbb{C})$, are they necessarily conjugate in $\textrm{M}_n(\R)$ as well?}

\textit{Answer.}
Yes. They have the same rational canonical form whose access depends no condition on the base field.

\miquestion {\color{blue} Give the $4 \times 4$ Jordan forms with minimal polynomial $(x - 1)(x - 2)^2$.}

\textit{Solution.} Let $J_n(\lambda)$ denote the Jordan block for some eigenvalue $\lambda$ of size $n\times n$. Then the form is
$$
\operatorname{diag}\{J_2(2),1,1\},\quad \text{ or } \operatorname{diag}\{J_2(2),2,1\}.
$$

\miquestion {\color{blue} Talk about Jordan canonical form. What happens when the field is not algebraically closed?}

\textit{Answer.}
Jordan canonical form is a diagonally blocked matrix consisting of $J_{n_i}(\lambda_i)$, where $\lambda_i$ is one of the eigenvalues.

If the base field is not algebraically closed, the characteristic polynomial may not split completely, i.e., some eigenvalue does not lie in the field. So the matrix is possibly not diagonalizable. The existence of Jordan canonical form requires that the field admits an extension to a field that contains all eigenvalues.

\miquestion {\color{blue} What are all the matrices that commute with a given Jordan block?}

\textit{Answer.} This can be proved by induction: only polynomials of this Jordan block itself.

\miquestion {\color{blue} How do you determine the number and sizes of the blocks for Jordan canonical form?}

\textit{Answer.} The number of blocks is not less than the number of distinct eigenvalues. The exponent of an eigenvalue in the minimal polynomial is exactly the maximal size of the block for this eigenvalue. The minimal polynomial and the invariant factors helps a lot. But all these methods cannot give an explicit solution without further computation.

\miquestion {\color{blue} For any matrix $A$ over the complex numbers, can you solve $B^2 = A$?}

\textit{Answer.} This can be done over any algebraically closed field (or somehow a sufficiently large field that allows the computation for Jordan canonical forms). The recipe is to reduce $A$ into the Jordan canonical form. For each Jordan block, say $J(\lambda)$, and for some analytically nice function, say $f:\GL_n(\C)\to \GL_n(\C)$, we obtain
$$
f(J)=\begin{pmatrix}
f(\lambda) & f'(\lambda) & \frac{1}{2!}f^{(2)}(\lambda) & \cdots & \frac{1}{(n-1)!}f^{(n-1)}(\lambda)\\
& f(\lambda) & f'(\lambda) & \ddots & \frac{1}{(n-2)!}f^{(n-2)}(\lambda)\\
& & \ddots & \ddots & \vdots \\
& & & \ddots & f'(\lambda) \\
& & & & f(\lambda)
\end{pmatrix}.
$$

\miquestion {\color{blue} What is rational canonical form?}

\textit{Answer.} Given any matrix $A$, we can find its invariant factors, say $f_1\mid \cdots \mid f_n$ such that there product is the characteristic polynomial $\operatorname{char}\operatorname{poly}(A)$, and $f_n=m(A)$, the minimal polynomial. The rational canonical form is the diagonal block of companion matrices of these.

\miquestion {\color{blue} Describe all the conjugacy classes of $3 \times 3$ matrices with rational entries which satisfy the equation $A^4 - A^3 - A + 1 = 0$. Give a representative in each class.}

\textit{Answer.} By assumption we have $(A - 1)^2 (A^2 + A + 1)=0$. It forces that $\operatorname{char}\operatorname{poly}(A)=(x-1)(x^2 + x + 1)$. So the invariant factors are uniquely determined, and the conjugacy class is unique. Its representative in canonical form is written as
$$
\begin{pmatrix}
    1 & 0 & 0\\
    0 & 0 & -1\\
    0 & 1 & 1\\
\end{pmatrix}.
$$

\miquestion {\color{blue} What $3 \times 3$ matrices over the rationals (up to similarity) satisfy $f(A) = 0$, where $f(x) = (x^2 + 2)(x - 1)^3$? List all possible rational forms.}

\textit{Solution.} By Cayley--Hamilton theorem, $\operatorname{char}\operatorname{poly}(A)\mid f$, and it is of degree 3. Then $\operatorname{char}\operatorname{poly}(A)=(x^2+2)(x-1)$ or $(x - 1)^3$ because $(x^2+2)$ is irreducible. The possibilities for invariant factors are 
$$
(x^2+2)(x-1)\quad \text{or}\quad x-1,x-1,x-1\quad \text{or}\quad x-1,(x-1)^2\quad \text{or}\quad (x-1)^3.
$$ The respective rational canonical forms are in the following:
$$
\begin{pmatrix}
    1 & 0 & 0\\
    0 & 0 & -2\\
    0 & 1 & 0\\
\end{pmatrix}, \quad
\begin{pmatrix}
    1 & 0 & 0\\
    0 & 1 & 0\\
    0 & 0 & 1\\
\end{pmatrix}, \quad
\begin{pmatrix}
    1 & 0 & 0\\
    0 & 0 & -1\\
    0 & 1 & 2\\
\end{pmatrix}, \quad
\begin{pmatrix}
    0 & 0 & 1\\
    1 & 0 & -3\\
    0 & 1 & 3\\
\end{pmatrix}.
$$

\miquestion {\color{blue} What can you say about matrices that satisfy a given polynomial (over an algebraically closed field)? How many of them are there? What about over a finite field? How many such matrices are there then?}

\textit{Answer.} By Cayley--Hamilton theorem that holds over all commutative rings, the minimal polynomial of the matrices are factors of the given polynomial $f$. Yet $f$ has nothing to do with the characteristic polynomial. Over an algebraically closed field, each Jordan normal form can be reached by a conjugation from the original matrix $A$ such that $f(A)=0$. So there are only finitely many types of Jordan normal forms (whereas there is possibly infinitely many $A$'s satisfying $f(A)=0$). This number is morally bounded by the number of partitions of $n=\deg f$.

Over the finite fields, the counting argument is essentially related to the number of conjugacy classes of $\GL_n(\F_q)$, which equals to the number of irreducible representations. 


\miquestion {\color{blue} What is a nilpotent matrix?}

\textit{Answer.}
Some $A$ such that $A^n=0$ for sufficiently large $n$.

\miquestion {\color{blue} When do the powers of a matrix tend to zero?}

\textit{Answer.} Consider this on a sufficiently large field such that we can talk about Jordan canonical forms. It turns out that any eigenvalue $\lambda$ should have norm $N(\lambda)<1$.

\miquestion {\color{blue} If the traces of all powers of a matrix $A$ are $0$, what can you say about $A$?}

\textit{Claim.} $A$ is nilpotent over fields of characteristic 0.

\textit{Proof.} Note that
$$
\forall k\in \mathbb{N}, \quad \operatorname{tr}(A^k)=0 \quad \Longleftrightarrow \quad \forall k\in \mathbb{N}, \quad F_k:=\lambda_1^k+\cdots+\lambda_n^k=0.
$$
As a symmetric monic polynomial, $\operatorname{char}\operatorname{poly}(A)=(\lambda-\lambda_1)\cdots(\lambda-\lambda_n)$ has non-leading coefficients whose forms are polynomials in $F_k$ by {\color{violet}\textbf{Newton Identity}}. Then $\operatorname{char}\operatorname{poly}(A)=\lambda^N$ and hence $A^N=0$.

\miquestion {\color{blue} When and how can we solve the matrix equation $\exp(A) = B$? Do it over the complex numbers and over the real numbers.}

{\color{violet}
\textit{Proposition.} We have the identity for any matrix over any field: $\det(\exp(A))=\exp(\tr(A))$.}

\textit{Answer.} Due to the identity, we see $\exp:\mathrm{M}_{n\times n}(F)\to \GL_n(F)$ is surjective for $F=\C$, but is not surjective for $F=\R$. When $B$ is a real matrix, it is required to have a positive determinant.  




\miquestion {\color{blue} Say we can find a matrix $A$ such that $\exp(A) = B$ for $B$ in $\operatorname{SL}_n(\R)$. Does $A$ also have to be in $\operatorname{SL}_n(\R)$? Can you take $A$ to be in $\operatorname{SL}_n(\R)$?}

\textit{Answer.} According to the formula in Question 17 above, we see $\det(B)=1=\exp(\tr(A))$, and hence $\tr(A)=0$. But this not necessarily implies that $A\in \SL_n(\R)$. But we can choose $A$ to be so, because $\SL_n(\R)$ is generated by its Lie algebra $\mathfrak{sl}_n(\R)$ via $\exp$, whose elements consists of matrices of trace 0.

\miquestion {\color{blue} Is a square matrix always similar to its transpose?}

\textit{Answer.} Yes. The same characteristic polynomial deduces the same invariant factors, and hence the same rational canonical form.

\miquestion {\color{blue} What are the conjugacy classes of $\operatorname{SL}_2(\R)$?}

\textit{Solution.} Since $\R$ is not algebraically closed, any characteristic polynomial of degree 2 has one of the following forms: $x^2+ax+b$, $(x-a)^2$. Note that in $\SL_2(\R)$, the constant term of characteristic polynomial is the determinant, which is 1. So there are 3 types of invariant factors: $x^2+ax+1$; $x-1,x-1$; $x+1,x+1$. Here comes two types of rational canonical forms:
$$
\begin{pmatrix}
0 & -1 \\
1 & -a
\end{pmatrix}, \quad
\begin{pmatrix}
1 & 0 \\
0 & 1
\end{pmatrix}, \quad
\begin{pmatrix}
-1 & 0 \\
0 & -1
\end{pmatrix},\quad a\in \R.
$$
Note that there are infinitely many conjugacy classes of $\SL_2(\R)$ when $a$ varies.

\miquestion {\color{blue} What are the conjugacy classes in $\operatorname{GL}_2(\mathbb{C})$?}

\textit{Solution.} Since $\C$ is algebraically closed, any characteristic polynomial of degree 2 splits completely as $(x-a)(x-b)$. The invariant factors are $x-a,x-a$ or $x^2-c x-d$. Here comes 2 types of conjugacy classes whose rational canonical forms are:
$$
\begin{pmatrix}
a & 0 \\
0 & a
\end{pmatrix}, \quad
\begin{pmatrix}
0 & d \\
1 & c
\end{pmatrix},\quad a,c,d\in \C.
$$

\end{questions}

\end{document}

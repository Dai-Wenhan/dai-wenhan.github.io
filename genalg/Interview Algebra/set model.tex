\documentclass{mathproblems}
\usepackage[all]{xy}
\usepackage{stmaryrd}
\usepackage{amsmath}
\usepackage{amssymb,mathrsfs}
\usepackage{amsthm}
\usepackage{xcolor}
\usepackage{multirow}
\usepackage{tikz}
\usepackage{tikz-cd}
\usetikzlibrary{cd,shapes,arrows,positioning,calc}

\usepackage{enumerate}
\usepackage{mathrsfs}
\usepackage{enumitem}
\usepackage{hyperref}
\usepackage{bm}
\usepackage{pgfplots}

\usepackage{geometry}
\geometry{left=2.5cm,right=2.5cm,bottom=3.4cm}


\newcommand\V{\mathcal{V}}
\newcommand\tD{\tilde{\mathcal{D}}}
\newcommand\Q{\mathbb{Q}}
\newcommand\bS{\mathbb{S}}
\newcommand\cH{\mathcal{H}}
\newcommand\FF{\mathscr{F}}
\newcommand\cF{\mathcal{F}}
\newcommand\A{\mathcal{A}}
\newcommand\B{\mathcal{B}}
\newcommand\R{\mathbb{R}}
\newcommand\C{\mathbb{C}}

\newcommand\bD{\mathbb{D}}
\newcommand\Fil{\mathrm{Fil}}
\newcommand\pDiv{p\mathsf{Div}}

\newcommand\T{\mathcal{T}}
\newcommand\Z{\mathbb{Z}}
\newcommand\N{\mathbb{N}}
\newcommand\NN{\mathcal{N}}
\newcommand\PP{\mathcal{P}}
\newcommand\QQ{\mathfrak{Q}}
\newcommand\D{\mathcal{D}}
\newcommand\OO{\mathcal{O}}
\newcommand\F{\mathbb{F}}

\newcommand\p{\mathfrak{p}}
\newcommand\g{\mathfrak{g}}
\newcommand\h{\mathfrak{h}}
\newcommand\q{\mathfrak{q}}
\newcommand\fC{\mathfrak{C}}
\newcommand\fa{\mathfrak{a}}
\newcommand\fb{\mathfrak{b}}
\newcommand\fc{\mathfrak{c}}
\newcommand\fd{\mathfrak{d}}
\newcommand\m{\mathfrak{m}}
\newcommand\al{\alpha}
\newcommand\pel{(\mathcal{O},\ast,\Lambda,\langle\cdot,\cdot\rangle,h)}
\newcommand\lm{\Lambda}
\newcommand\om{\Omega}
\newcommand\gm{\Gamma}
\newcommand\hZ{\widehat{\Z}}
\newcommand\LL{\mathcal{L}}
\newcommand\Crys{\mathsf{Crys}}
\newcommand\dR{\mathrm{dR}}
\newcommand\Qb{\overline{\mathbb{Q}}}
\newcommand\Qpb{\overline{\mathbb{Q}}_{p}}
\newcommand\Qp{\mathbb{Q}_{p}}
\newcommand\Qlb{\overline{\mathbb{Q}}_{\ell}}
\newcommand\Fb{\overline{\mathbb{F}}}
\newcommand\Fpb{\overline{\mathbb{F}}_{p}}
\newcommand\Fqb{\overline{\mathbb{F}}_{q}}
\newcommand\Zb{\overline{\mathbb{Z}}}
\newcommand\rb{\overline{\rho}}
\newcommand\der{\mathrm{der}}
\newcommand\GO{\mathrm{GO}}
\newcommand\GU{\mathrm{GU}}
\newcommand\GL{\mathrm{GL}}
\newcommand\Frob{\mathrm{Frob}}

\newcommand\alg{\mathrm{alg}}
\newcommand\un{\mathrm{un}}
\newcommand\PSL{\mathrm{PSL}}
\newcommand\PSO{\mathrm{PSO}}
\newcommand\PGL{\mathrm{PGL}}
\newcommand\SL{\mathrm{SL}}
\newcommand\SU{\mathrm{SU}}
\newcommand\SO{\mathrm{SO}}
\newcommand\Sp{\mathrm{Sp}}
\newcommand\GSp{\mathrm{GSp}}
\newcommand\Spin{\mathrm{Spin}}
\newcommand\simto{\stackrel{\sim}{\longrightarrow}}
\newcommand\simeqto{\stackrel{\simeq}{\longrightarrow}}
\newcommand\congto{\stackrel{\cong}{\longrightarrow}}
\renewcommand\P{\mathbb{P}}



\newcommand\Rep{\mathsf{Rep}}
\newcommand\Sch{\mathsf{Sch}}
\newcommand\PR{\mathrm{PR}}
\newcommand\DP{\mathrm{DP}}
\newcommand\Ra{\mathrm{Ra}}
\newcommand\Hasse{\mathrm{Hasse}}
\newcommand\cris{\mathrm{cris}}
\newcommand\ad{\mathrm{ad}}
\newcommand\Ad{\mathrm{Ad}}
\newcommand\univ{\mathrm{univ}}
\DeclareMathOperator{\Ext}{Ext}
\DeclareMathOperator{\Spl}{Spl}
\DeclareMathOperator{\Tr}{Tr}
\DeclareMathOperator{\tr}{tr}
\DeclareMathOperator{\Cl}{Cl}
\DeclareMathOperator{\Pic}{Pic}
\DeclareMathOperator{\rad}{Rad}
\DeclareMathOperator{\Hom}{Hom}
\DeclareMathOperator{\Spec}{Spec}
\DeclareMathOperator{\Aut}{Aut}
\DeclareMathOperator{\Nm}{Nm}
\DeclareMathOperator{\Ann}{Ann}
\DeclareMathOperator{\St}{St}
\DeclareMathOperator{\ev}{ev}

\DeclareMathOperator{\Ker}{Ker}
\DeclareMathOperator{\Coker}{Coker}
\DeclareMathOperator{\im}{im}

\DeclareMathOperator{\Id}{Id}
\DeclareMathOperator{\Ind}{Ind}
\DeclareMathOperator{\Res}{Res}
\DeclareMathOperator{\Gal}{Gal}
\DeclareMathOperator{\End}{End}
\DeclareMathOperator{\ord}{ord}
\DeclareMathOperator{\Sym}{Sym}
\DeclareMathOperator{\disc}{disc}
\DeclareMathOperator{\rank}{rank}
\DeclareMathOperator{\Lie}{Lie}

\DeclareMathOperator{\Sht}{Sht}
\DeclareMathOperator{\Bun}{Bun}

\DeclareMathOperator{\Gr}{Gr}

\course{Interview Questions on Algebra}
\studentname{Wenhan Dai}



\begin{document}

\centerline{\textbf{Set IX: Categories and Functors}}

\begin{questions}
\miquestion {\color{blue} Which is the connection between $\Hom$ and tensor product? What is this called in representation theory?}

\textit{Answer.} They are a pair of adjoint functors, i.e., in some small category $\mathcal{C}$, 
$$
\Hom_{\mathcal{C}}(X\otimes Y, Z)\longrightarrow \Hom_{\mathcal{C}}(X,\Hom(Y,Z)).
$$
This is called {\color{violet} \textbf{Frobenius reciprocity}} in representation theory, which states tensor product as the functor for induced representations and $\Hom$ as the functor of restrictions, respectively.

\miquestion {\color{blue} Can you get a long exact sequence from a short exact sequence of abelian groups together with another abelian group?}

\textit{Answer.} This is just the long exact sequence of group cohomology. For a (discrete) group $G$, which is not necessarily abelian, acting on another abelian group $M$ (with discrete topology), which is called a $G$-module, we can define $H^i(G,M):=\Ext_{\Z[G]}^i(\Z,M)$. This theory is covariant in $M$ and contravariant in $G$. 

\miquestion {\color{blue} Do you know what the Ext functor of an abelian group is? Do you know where it appears? What is $\Ext(\Z/m\Z,\Z/n\Z)$? What is $\Ext(\Z/m\Z,\Z)$? How about $\Ext(\Z/m\Z,\Q)$?}

\textit{Answer.} The Ext functor is a derived functor that measures the failure of a short exact sequence of modules to split. Also, $\Ext(A,B)$ classifies abelian extensions of $A$ by $B$. It appears in Cartan--Eilenberg's 1956 book \textit{Homological Algebra}.

\textit{Solution.} For any $\Z$-module $M$, the homomorphism $\Z\to M$ is defined by the image of 1 in $H$. So 
$$
\Hom(\Z,M)=M,\quad \Ext(\Z,M)=0
$$
because $\Z$ is a projective module (c.f. Set 7, Question 8). We begin the computation with a projective resolution of $\Z/m\Z$ as follows:
$$
0\longrightarrow \Z \stackrel{\times m}{\longrightarrow} \Z \longrightarrow \Z/m\Z\longrightarrow 0.
$$
Taking the contravariant functor $\Hom(-,M)$, we get \vspace{-4pt}
\begin{center}
\begin{tikzcd}[row sep=6pt, column sep=normal]
& & M \ar[d,"="{rotate=-90,description,near start}] & M \ar[d,"="{rotate=-90,description,near start}]\\
0 \ar[r] & \Hom(\Z/m\Z,M) \ar[r] & \Hom(\Z,M) \ar[r,"\times m"] & \Hom(\Z,M) \ar[lld,out=-5,in=175,overlay] \\
& \Ext(\Z/m\Z,M) \ar[r] & \Ext(\Z,M)=0.
\end{tikzcd}
\end{center}\vspace{-4pt}
It follows that
$$
\Hom(\Z/m\Z,M)=\Ker m=M[m],\quad \Ext(\Z/m\Z,M)=M/mM.
$$
In particular, for $d=(m,n)$, $\Hom(\Z/m\Z,\Z)=0$ and $\Hom(\Z/m\Z,\Z/n\Z)=\Z/d\Z$. Moreover,
$$
\Ext(\Z/m\Z,\Z/n\Z)=\Z/d\Z,\quad \Ext(\Z/m\Z,\Z)=\Z/m\Z.
$$
Also, we have $\Hom(\Z/m\Z,\Q)=\Ext(\Z/m\Z,\Q)=0$.





\end{questions}
\end{document}
